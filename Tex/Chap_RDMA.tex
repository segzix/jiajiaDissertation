\chapter{RDMA 通信栈的设计与实现}\label{chap:RDMA}{
    本章主要介绍 M-JIAJIA系统中 RDMA 通信栈的架构设计与实现方案。作为通信层组件,RDMA 通信栈与 UDP 通信栈均可完成系统消息通信任务,用户可根据底层通信环境在系统运行前通过 .jiaconf 配置实现通信栈的动态切换。
    在架构设计上,RDMA 和 UDP 通信栈均采用多线程设计,但由于 RDMA通信栈与底层网络适配器深度集成,在实现上与 UDP 通信栈呈现显著差异。

    RDMA 通信首先需要根据应用场景选择合适的通信模式和通信原语,并基于所选模式和原语构建 RDMA 通信管理器,以管理所需的通信资源。随后,根据通信链路模式执行建连后通信或无连接通信。

    \section{设计概览}\label{sec:RDMA设计概览}

    \subsection{RDMA 通信模式与通信原语选择}

    M-JIAJIA 在可靠连接(RC)模式下采用 Send 原语进行消息传递,这一设计选择主要原因是M-JIAJIA 对通信可靠性要求严格。
    RC 模式免除了应用层的分片处理,同时还具有可靠性与有序性保障,从而大大简化了系统设计。

    \subsection{RDMA 通信管理器设计}

    M-JIAJIA 的 RDMA 通信管理器(jia\_context\_t)旨在管理 RDMA 通信所涉及的资源和配置参数。具体可分为以下几部分:

    \begin{enumerate}[label=\arabic*., leftmargin=1em, align=left]
        \item \textbf{RDMA 设备相关信息}
              \begin{itemize}
                  \item \textbf{设备上下文}:设备上下文是应用程序与 RDMA 硬件之间的接口,用于管理所有其他 RDMA 资源。
                  \item \textbf{设备端口号}:设备端口号用于标识 RDMA 设备进行通信的物理端口。
              \end{itemize}

        \item \textbf{RDMA 资源管理信息}
              \begin{itemize}
                  \item \textbf{保护域}:保护域(Protect Domain, PD)是一种资源隔离手段,在RC 模式下用于限制 QPs 可以访问内存区域(Memory Region,MR)的范围。
                  \item \textbf{内存区域}:内存区域是应用程序注册后可被 RDMA 设备直接访问的内存。M-JIAJIA 采用 Send 原语进行通信,因此需分别为发送和接收准备独立的内存区域。在多通信场景下,考虑到输入的不确定性,M-JIAJIA 采用单输出、多输入的内存区域设计方案。
              \end{itemize}

        \item \textbf{RDMA 事件通知信息}
              \begin{itemize}
                  \item \textbf{完成事件通道}:完成事件通道提供 CQ 事件通知机制,用于通知工作请求是否完成。M-JIAJIA 提供独立的发送和接收完成事件通道。
              \end{itemize}

        \item \textbf{RDMA 连接管理信息}
              \begin{itemize}
                  \item \textbf{RDMA 连接管理结构}:该结构用于管理与每个连接相关的信息。包括连接状态、发送序号和接收序号、RDMA 连接管理标识符(rdma\_cm\_id)、输入队列和输入内存区域。
                  \item \textbf{RDMA 连接管理结构数组}:连接管理结构数组记录了与每个远端节点的连接信息。
                  \item \textbf{RDMA 建立连接线程标识}:M-JIAJIA 采用客户端-服务器模式建立连接。在建连阶段,节点分别创建客户端和服务器线程以完成连接。其建连方案具有独特性:每个节点至多创建一个服务器线程用于监听连接请求,同时可创建多个客户端线程主动发起连接。线程标识用于存储和管理线程的信息,便于线程创建、同步和管理。
              \end{itemize}
    \end{enumerate}

    \section{可靠通信连接建立机制}\label{sec:可靠通信连接建立机制}

    为构建基于 RDMA 可靠连接(RC)的通信系统,RDMA 可靠通信的建立分为以下四个主要阶段:

    \begin{enumerate}[label=\arabic*.]
        \item 初始化上下文。包括获取设备列表、选择硬件设备、打开设备获取上下文、配置物理端口以及输入/输出消息队列初始化等步骤。
        \item 建立连接。通过创建线程向其他主机发起建连请求、并响应其他主机的连接请求,完成任意两节点之间的建连。
        \item 初始化通信资源。主要用于注册输入/输出内存区域(MR)。
        \item 多线程通信。创建线程分别负责下发发送/接收工作请求(WR),同时创建专门处理消息的线程,以优化通信效率。
    \end{enumerate}

    如图~\ref{fig:mjiajia-cm-connection}所示,建连过程需要客户端和服务器紧密配合,共同完成一系列复杂的步骤,最终生成用于通信的 rdma\_cm\_id 。

    \begin{figure}[H]
        \centering
        \includegraphics[width=\textwidth]{Img/rdma_init.drawio.pdf}
        \bicaption{M-JIAJIA RDMA CM 建连流程图}{M-JIAJIA RDMA CM Connection Establishment Flowchart}
        \label{fig:mjiajia-cm-connection}
    \end{figure}

    \subsection{RDMA 集群建连算法}

    M-JIAJIA 运行时要求任意两节点具备通信能力,因此必须在任意两节点之间至少建立一个连接。
    M-JIAJIA 采用客户端服务器模式,通过创建多个客户线程与服务线程去建立连接,并采用了算法~\ref{alg:connection-algo}所示的建连算法,
    不必为每个连接设立单独的服务器处理。

    在rdma同步的过程中,主机号偏小的主机会创建多个client线程与主机号偏大的主机server线程尝试建立连接;
    同时除jia\_pid为0外所有的主机都会创建一个server线程来监听建立的rdma连接,最终每两台主机之间建立一个rdma连接;
    如图~\ref{fig:RDMA-connection-build}所示,创建线程并建立连接。

    \begin{figure}[H]
        \centering
        \includegraphics[width=0.5\textwidth]{Img/rdma_sync.drawio.pdf}
        \bicaption{\enspace RDMA集群建连示例}{\enspace RDMA Cluster Connection Establishment Example}
        \label{fig:RDMA-connection-build}
    \end{figure}

    \begin{algorithm}
        \caption{RDMA Cluster Connection Establishment Algorithm}\label{alg:connection-algo}
        \begin{algorithmic}[1]
            \Procedure{sync\_connection}{$hosts$, $pid$}
            \If{$pid$ $\neq$ 0}
            \State \Call{pthread\_create}{\&ctx.server\_thread, NULL, server\_thread, \&ctx}
            \EndIf
            \State $num\_clients$ $\gets$ $hosts$ - $pid$ - 1
            \If{$num\_clients$ $>$ 0}
            \State $ctx.client\_threads$ $\gets$ \Call{malloc}{$num\_clients$ * sizeof(pthread\_t)}
            \For{$i \gets 0$ to $num\_clients$}
            \State $target\_hosts$ $\gets$ \Call{malloc}{sizeof(int)}
            \State $*target\_hosts$ $\gets$ $pid + i + 1$
            \State \Call{pthread\_create}{\&ctx.client\_threads[i], NULL, client\_thread, target\_hosts}
            \EndFor
            \EndIf
            \If{$pid$ $\neq$ 0}
            \State \Call{pthread\_join}{ctx.server\_thread, NULL}
            \EndIf

            \If{$num\_clients$ > 0}
            \For{$i \gets 0$ to $num\_clients$}
            \State \Call{pthread\_join}{ctx.client\_threads[i], NULL}
            \EndFor
            \EndIf
            \EndProcedure
        \end{algorithmic}
    \end{algorithm}

    \section{RDMA 多线程通信架构设计}\label{sec:RDMA 多线程通信架构设计}

    RDMA通信栈与UDP通信栈类似的架构,将通信任务划分为发送、接收和处理三个子任务,并为每个子任务分配独立线程执行,以提高并发效率。
    见图~\ref{fig:RDMA-communication-design}。

    \begin{figure}[H]
        \centering
        \includegraphics[width=1.0\textwidth]{Img/RDMA_design.drawio.pdf}
        \bicaption{\enspace RDMA通信设计}{RDMA communication design}
        \label{fig:RDMA-communication-design}
    \end{figure}


    \subsection{RDMA 多线程任务划分}

    \subsubsection{RDMA通信接收端线程任务划分}
    RDMA通信接收端由listen与server两个线程组成,他们的任务划分如下:

    \begin{itemize}
        \item \textbf{listen thread}:

              listen thread 轮询或事件通知机制检测Recv操作的完成状态,检测到Recv操作完成时通知server线程进行处理;

              listen thread 监听rdma事件主要分为以下几个步骤:
              \begin{enumerate}[leftmargin=*, nosep]
                  \item 调用ibv\_get\_cq\_event函数监听recv\_channel通道;
                  \item 监听事件成功后递减原子变量post\_value,递增原子变量busy\_value,并更新tail指针,rcv\_seq记录序号;
                  \item 完毕后确认rdma事件。
              \end{enumerate}

              listen thread的伪代码如算法~\ref{alg:listenRDMA-thread}所示。
              \begin{algorithm}
                  \caption{listen thread algorithm}\label{alg:listenRDMA-thread}
                  \begin{algorithmic}[1] % [1] 使得每行都有行        
                      \Procedure{ListenThread}{}
                      \State \Call{post\_recv}{$ $}
                      \State \textbf{return} NULL
                      \EndProcedure

                      \Function{post\_recv}{}
                      \While{$true$}
                      \State $\{ cq\_ptr, context\} \gets$ \Call{ibv\_get\_cq\_event}{$recv\_comp\_channel$}
                      \State $inqueue \gets (rdma\_connect\_t)context.inqueue$
                      \State \Call{ibv\_req\_notify\_cq}{$cq\_ptr$}
                      \State $wc \gets$ \Call{ibv\_poll\_cq}{$cq\_ptr$}

                      \State
                      \If{$wc.status \neq \text{IBV\_WC\_SUCCESS}$}
                      \State \Call{log\_err}{$post\_recv \quad error$}
                      \Else
                      \State \Call{atomic\_fetch\_sub}{$inqueue$->$post\_value$}
                      \State \Call{atomic\_fetch\_add}{$inqueue$->$busy\_value$}

                      \State
                      \State $connect\_array[inqueue$->$queue[inqueue$->$tail]$->$topid].rcv\_seq $++
                      \State $inqueue$->$tail \gets (inqueue$->$tail$+1$) \% QueueSize$
                      \EndIf
                      \EndWhile

                      \State
                      \State \Call{ibv\_ack\_cq\_events}{$cq\_ptr$}
                      \State \textbf{return}
                      \EndFunction
                  \end{algorithmic}
              \end{algorithm}
        \item \textbf{server thread}:

              初始化时,预先为每个QP下发多个Recv Buffer用于预备接收;
              在得到listen thread通知后,从inqueue中取出消息进行处理,同时继续下发新的Recv请求,
              确保始终有足够的Recv Buffer接收后续的消息;

              server thread 监听rdma事件主要分为以下几个步骤:
              \begin{enumerate}[leftmargin=*, nosep]
                  \item 通过busy\_value判断是否有消息需要进行处理;
                  \item 确定处理成功消息后递减原子变量busy\_value,递增原子变量free\_value,并更新head指针;
                  \item 调用check\_flags函数看是否有空位需要重新下发recv\_wr。
              \end{enumerate}

              server thread的伪代码如算法~\ref{alg:serverRDMA-thread}。
              \begin{algorithm}
                  \caption{server thread algorithm}\label{alg:serverRDMA-thread}
                  \begin{algorithmic}[1] % [1] 使得每行都有行号
                      \Procedure{ServerThread}{}
                      \State \Call{init\_post\_recv\_wr}{$ $}

                      \While{$true$}
                      \For{$i \gets 0$ to $hostc$}
                      \State $tmp\_connect \gets connect\_array[i]$
                      \State $in\_queue \gets tmp\_connect$->$inqueue$
                      \If{\Call{atomic\_load}{$tmp\_connect.inqueue$->$busy\_value$}}
                      \State \Call{msg\_handle}{$inqueue$->$queue[head]$}
                      \State \Call{atomic\_fetch\_sub}{$inqueue$->$busy\_value$}
                      \State \Call{atomic\_fetch\_add}{$inqueue$->$free\_value$}
                      \If{$i == jia\_pid$}
                      \State $tmp\_connect.rcv\_seq$ ++
                      \EndIf
                      \State $inqueue$->$head \gets (inqueue$->$head$+1$) \% QueueSize$
                      \EndIf
                      \EndFor

                      \If{$i \neq jia\_pid$}
                      \State \Call{check\_flags}{$i$}
                      \EndIf
                      \EndWhile
                      \State \textbf{return}
                      \EndProcedure
                  \end{algorithmic}
              \end{algorithm}
    \end{itemize}

    在接收端inqueue的每个msg位置,均有三种状态可以进行转换(post,free,busy),状态的转换与线程的作用见图~\ref{fig:RDMA-recv-state}。
    \begin{figure}[H]
        \centering
        \includegraphics[width=0.6\textwidth]{Img/recv_state.drawio.pdf}
        \bicaption{RDMA接收端inqueue中msg的状态转换}{RDMA recv state change}
        \label{fig:RDMA-recv-state}
    \end{figure}

    \subsubsection{RDMA通信发送端线程任务划分}
    RDMA通信接收端由main/server两个入队线程,和一个client发送线程组成,他们的任务划分如下:

    \begin{itemize}
        \item \textbf{main/server thread}:

              用于下发rdma需要发送的msg,并将msg放入outqueue队列中,
              队列中的每个消息包含目标主机的IP地址和消息内容;
        \item \textbf{client thread}:

              将对应的msg从outqueue队列中取出,
              并调用\texttt{ibv\_post\_send} API发送msg给对应主机;
              发送后发送端通过轮询或事件通知机制(如Completion Queue)检测Send操作的完成状态,操作失败则记录错误日志。

              client thread rdma发送消息主要分为以下几个步骤:
              \begin{enumerate}[leftmargin=*, nosep]
                  \item 通过信号量同步从inqueue中取出消息;
                  \item 调用post\_send函数发送给对应的主机,并通过ack确认是否发送成功;
                  \item 发生消息成功后更新head指针,并用snd\_seq记录序号;
              \end{enumerate}

              \newpage
              client thread的伪代码如算法~\ref{alg:clientRDMA-thread}所示。
              \begin{algorithm}
                  \caption{client thread algorithm}\label{alg:clientRDMA-thread}
                  \begin{algorithmic}[1] % [1] 使得每行都有行号
                      \Procedure{ClientThread}{}
                      \While{$true$}
                      \State \Call{sem\_wait}{$outqueue.busy\_count$}
                      \State $msg\_ptr \gets$ \Call{dequeue}{$outqueue$}
                      \State $msg\_ptr$->$seqno \gets connect\_array[msg\_ptr$->$to\_pid]$

                      \State
                      \If{$msg\_ptr$->$topid$ = $jia\_pid$}
                      \State \Call{memcpy}{$inqueue[tail],msg\_ptr$}
                      \State \Call{atomic\_fetch\_sub}{$inqueue$->$free\_value$}
                      \State \Call{atomic\_fetch\_add}{$inqueue$->$busy\_value$}
                      \State $inqueue$->$tail \gets (inqueue$->$tail$+1$) \%QueueSize$
                      \Else
                      \While{\Call{post\_send}{$connect\_array[msg\_ptr$->$to\_pid]$}}
                      \EndWhile
                      \EndIf

                      \State
                      \State $connect\_array[msg\_ptr$->$topid].snd\_seq$++
                      \State $inqueue$->$head \gets (inqueue$->$head$+1$) \% QueueSize$
                      \State \Call{sem\_post}{$outqueue.free\_count$}
                      \EndWhile
                      \State \textbf{return}
                      \EndProcedure
                  \end{algorithmic}
              \end{algorithm}
    \end{itemize}

    在发送端inqueue的每个msg位置,均有两种状态可以进行转换(free,busy),状态的转换与线程的作用见图~\ref{fig:RDMA-send-state}。
    \begin{figure}[H]
        \centering
        \includegraphics[width=0.6\textwidth]{Img/send_state.drawio.pdf}
        \bicaption{RDMA发送端outqueue中msg的状态转换}{RDMA send state change}
        \label{fig:RDMA-send-state}
    \end{figure}

    \section{本章小结}
    本章 \ref{sec:RDMA设计概览} 节介绍了 M-JIAJIA RDMA通信模式与通信原语选择,并给出了 M-JIAJIA 中RDMA管理器的基本设计与组成模块。

    本章 \ref{sec:可靠通信连接建立机制} 节介绍了 M-JIAJIA RDMA基本的建连过程和对应的算法。

    本章 \ref{sec:RDMA 多线程通信架构设计} 节介绍了 M-JIAJIA 多线程的任务划分算法,以及接收端发送端每个msg slot的状态转换。
}