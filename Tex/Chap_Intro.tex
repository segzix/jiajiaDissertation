\chapter{绪论}\label{chap:introduction}{

  \section{本文研究背景与意义}
  随着信息技术的迅猛发展,计算密集型与数据密集型任务已成为人工智能、流媒体、大数据、云计算和科学计算等领域面临的双重挑战。
  为应对这些挑战,大规模分布式集群系统应运而生,要求采用多种并行编程模型,结合松耦合的任务调度与数据共享策略,以实现跨节点的负载均衡与协同执行。

  从并行编程角度看,主流并行范式主要基于三种模型:共享内存、消息传递和分布式共享内存(DSM),其特点如下:
  \begin{itemize}
    \item \textbf{共享内存模型}: 所有计算核心共享全局主存,通过各自的私有高速缓存(cache)存储数据副本,以加快访问速度;
    \item \textbf{消息传递模型}: 各节点进程拥有独立的虚拟地址空间,进程间通过显式通信原语(如 Send/Recv)进行数据交换;
    \item \textbf{分布式共享内存模型}: 将共享内存的理念扩展至分布式系统,在物理上分离的内存上构建虚拟共享内存抽象,底层通过消息传递机制实现数据共享。
  \end{itemize}

  在三种模型中,分布式共享内存(DSM)结合了共享内存的编程便利性与消息传递的系统可扩展性,具有以下优势:
  \begin{enumerate}[leftmargin=1em, align=left]
    \item \textbf{相较于消息传递:} DSM 提供透明的数据共享机制,抽象层与传统共享内存模型保持一致,显著降低了分布式程序开发的复杂度;
    \item \textbf{相较于共享内存架构:} DSM 可突破单节点内存容量限制,在构建大规模负载处理系统时展现出更优的扩展性与性价比。
  \end{enumerate}

  然而,DSM 仍面临两个核心挑战:高昂的跨节点通信延迟和复杂的数据一致性维护。
  随着高性能互连网络和远程直接内存访问(RDMA)技术的不断发展,为应对传统消息传递模型中的显式分区与通信管理难题提供了新思路。
  现代 DSM 系统不仅突破了单节点内存容量瓶颈,还在保障一致性的前提下支持大规模、高并发的数据处理。

  本文以国产软件 DSM 系统 JIAJIA\citep{huweiwu2001sma,huweiwu2024ca,1999huweiwuJIAJIA} 为研究对象。
  JIAJIA 是由胡伟武教授于 20 世纪 90 年代末自主研发的我国首个 SDSM 系统,采用页粒度共享与 NUMA 架构构建虚拟共享内存,通过缓存机制优化远程访问,节点间基于单线程 UDP 进行通信。
  该系统填补了国内在该领域的空白,已被全球百余家科研机构采用,并应用于国产高性能计算机中。

  在此基础上,M-JIAJIA 针对传统 JIAJIA 存在的问题进行了如下优化:
  \begin{enumerate}
    \item 设计并实现支持 UDP 与 RDMA 双通信栈的架构,有效缓解了传统软件 DSM 系统在网络性能上的瓶颈,并为高性能网络环境下的 DSM 发展提供新思路;
    \item 通过将通信过程划分为入队、发送、监听、接收和处理五个阶段,并结合多线程机制实现流水线式并行通信模型,从而提高通信并发度,充分释放底层硬件性能。
  \end{enumerate}

  \subsection{软件分布式共享内存系统发展历程}
  \begin{enumerate}[leftmargin=1em, align=left]
    \item \textbf{早期 DSM 系统在一致性模型方面的发展}
          \begin{itemize}
            \item \textbf{顺序一致性}:1986 年,Li Kai 博士在论文\citep{likai1986svm} 中首次验证了在分布式系统上构建共享虚拟内存(SVM)的可行性,并于 1988 年开发出首个软件 DSM 系统 IVY\citep{likai1988ivy}。该系统采用页粒度共享,支持 SWMR 模式,并通过顺序一致性维持数据一致性,在并行计算中取得良好效果。
            \item \textbf{释放一致性(急切更新)}:Munin\citep{bennett1990munin} 采用释放一致性(RC)模型,仅在同步点执行一致性操作,降低通信成本。它还引入数据类型相关一致性策略,并结合多写协议与延迟更新机制,有效缓解假共享问题。
            \item \textbf{释放一致性(懒惰更新)}:TreadMarks\citep{amza1996treadmarks} 提出懒惰释放一致性(LRC),仅在锁释放者与新获取者之间传递一致性信息,进一步降低了通信量。
            \item \textbf{域一致性}:JIAJIA 采用域一致性(SC),同步范围仅限于锁保护区域的修改操作,基于锁实现缓存一致性协议,以锁记录写通知,替代目录方案,降低协议复杂度并提升系统可扩展性。
          \end{itemize}

    \item \textbf{软件 DSM 研究热度下降}

          随着多核架构的普及,片内通信远快于基于网络的 DSM 通信,软件 DSM 优势减弱。研究重心转向优化一致性协议与减少通信开销\citep{Cheung1999AMP, abe2003movinghomedsm}。
          \begin{itemize}
            \item 研究趋势包括:
                  \begin{itemize}
                    \item 优化一致性协议以降低通信成本;
                    \item 减少数据传输量,如优化写通知与失效消息。
                  \end{itemize}
          \end{itemize}

    \item \textbf{FaRM:RDMA 在 DSM 中的首次应用}

          2014 年,微软推出 FaRM\citep{drago2014farm} 系统,首次将 RDMA 应用于 DSM 替代传统 TCP/IP 通信栈,显著降低延迟并提高吞吐,引发了后续相关研究浪潮,证明了 RDMA 与 DSM 结合的可行性。

    \item \textbf{RDMA 驱动的后续 DSM 系统}
          \begin{itemize}
            \item \textbf{DaRPC(2014)}:基于 RDMA 实现异步非阻塞 RPC,提高通信性能;
            \item \textbf{FaSST(2016)}:使用 RDMA Send/Recv 实现高吞吐事务处理;
            \item \textbf{Argodsm(2015)}:提出自失效与自降级机制,减少一致性维护开销,通信层基于 MPI;
            \item \textbf{Grappa(2015)}:采用多线程优化任务调度,支持数据密集型计算,底层仍基于 MPI;
            \item \textbf{GAM(2018)}:结合部分存储一致性(PSO)与 RDMA 多种传输语义,提升吞吐能力;
            \item \textbf{Drust(2024)}:融合 Rust 所有权模型约束读写顺序,降低一致性维护成本,兼容 SWMR 访问模式。
          \end{itemize}

  \end{enumerate}

  \subsection{软件 DSM 系统的机遇与挑战}
  面对新型硬件架构与不断演化的应用场景,现代软件 DSM 系统既迎来了前所未有的发展机遇,也面临重重挑战。

  \textbf{机遇:RDMA 高速互连网络。}RDMA 显著降低跨节点内存访问延迟,且技术持续迭代。
  例如,英伟达的 ConnectX-8 网卡已实现 800 Gbps 带宽,接近本地内存访问速度,甚至优于部分 NUMA 节点间互联(如 QPI),为软件 DSM 系统带来新的性能突破口。

  \textbf{挑战:一致性与性能之间的权衡。}尽管 RDMA 减少了通信延迟,但在软件 DSM 系统中实现严格一致性(如顺序一致性)仍需频繁同步,容易引发性能瓶颈。如何在保证一致性的前提下提升系统性能,依然是当前研究的关键问题。

  \section{本文研究内容与主要贡献}
  本文旨在构建一个支持传统共享内存编程模型抽象、同时融合新一代 RDMA 网络技术的软件 DSM 系统,以提升系统通信性能与可扩展性。

  具体贡献如下:
  \begin{enumerate}[leftmargin=1em, align=left]
    \item 设计并实现了一个支持双通信栈(UDP + RDMA)的软件 DSM 系统 M-JIAJIA,有效发挥 RDMA 高带宽优势;
    \item 引入多线程机制重构 JIAJIA 原有的单线程通信模型,通过流水线并发通信提高系统效率;
    \item 实现了远程预取机制,优化取页流程,显著降低通信次数,从而提升局部性强的程序性能;
    \item 对基于 \texttt{M-JIAJIA} 开发的应用程序进行了性能开销分析,并与 OpenMPI 的通信接口进行了对比测试,为相关研究提供参考。
  \end{enumerate}

  \section{本文组织结构}
  第\ref{chap:introduction}章介绍研究背景与意义,回顾软件 DSM 的发展历程、挑战与机遇,并概述本文的研究内容与主要贡献;

  第\ref{chap:sdsm}章介绍软件 DSM 与 RDMA 的技术基础,阐述 DSM 的核心机制、系统架构及 JIAJIA 系统的设计与一致性模型;

  第\ref{chap:MJIAJIA}章详细介绍 M-JIAJIA 系统的架构设计与实现细节;

  第\ref{chap:RDMA}章重点介绍 M-JIAJIA 中 RDMA 通信栈的设计与实现过程;

  第\ref{chap:experiments}章展示系统性能测试结果,并分析其性能瓶颈及优化效果;

  第\ref{chap:summary}章对全文进行总结,并展望 M-JIAJIA 系统未来的发展方向。
}