%---------------------------------------------------------------------------%
%->> Frontmatter
%---------------------------------------------------------------------------%
%-
%-> 生成封面
%-

\maketitle% 生成中文封面
\MAKETITLE% 生成英文封面
%-
%-> 作者声明
%-
\makedeclaration% 生成声明页
%-
%-> 中文摘要
%-
\intobmk\chapter*{摘\quad 要}% 显示在书签但不显示在目录
\setcounter{page}{1}% 开始页码
\pagenumbering{Roman}% 页码符号

软件分布式共享内存(SDSM)系统通过软件库在分布式系统上构建一个虚拟共享内存抽象层,为分布式应用开发提供类似于多线程的编程范式。
与消息传递模型相比,SDSM 提供的全局地址空间支持透明的跨节点内存访问,
使得开发人员能够直接操作指针相关的复杂数据结构,从而显著降低了分布式程序开发与调试的复杂性。
此外,与紧耦合的分布式共享内存系统不同,SDSM 以较低的成本提供了更大的共享内存空间。

然而,传统的 SDSM 系统长期受到通信延迟的限制。近年来,随着远程直接内存访问(RDMA)等高速网络技术的进步,
跨节点访存和节点内访存的时延差距逐渐缩小,促使 SDSM 系统在分布式计算和分布式存储领域重新成为研究热点。

基于国产软件分布式共享内存系统 JIAJIA,本文设计并实现了支持 UDP 和原生 RDMA 双通信栈的 M-JIAJIA 系统。
并结合现代多核服务器的硬件特性和多样化并行负载的资源需求,提出了以下优化方案:

\begin{enumerate}[label=\arabic*.]
    \item \textbf{网络技术优化}:引入RDMA网络技术,保证多主机之间高速的可靠通信,降低节点之间的通信开销;
    \item \textbf{通信任务并行化}:引入多线程协同机制,将通信任务进行划分,保证硬件资源的充分利用;
    \item \textbf{跨节点访存优化}:设计远程预取机制,在远程取页时预取相邻的页并赋予受限权限,进一步降低局部性强的应用的内存访问延迟;
    \item \textbf{可定制架构}:系统将通信层与系统任务分离,在保证性能的同时,通过模块化结构提升系统的易用性与可扩展性,支持更灵活地对系统进行配置。
\end{enumerate}

M-JIAJIA 系统在多个并行测试框架上的应用程序(如 SPLASH-2 中的 LU、Water;
NAS 中的 EP、IS)以及经典并行应用(如矩阵乘法 MM、圆周率计算 PI、旅行商问题 TSP)和方程求解(Jacobi、SOR)上进行了测试。
实验结果表明,在 RDMA 网络下,通信密集型程序(如 TSP、MM、Water、LU、SOR)的性能相较于传统以太网提升了 16\%-64\%。
在传统以太网上,远程预取机制对局部性强的应用(如 MM:41\%,LU:13\%,SOR:12\%)有显著提升,
而在 RDMA 网络下,由于通信延迟较低,预取机制对性能影响较小。

\keywords{软件分布式共享内存, JIAJIA, 多线程, 远程直接内存访问}% 中文关键词
%-
%-> 英文摘要
%-
\intobmk\chapter*{Abstract}% 显示在书签但不显示在目录

Software Distributed Shared Memory (SDSM) systems create a virtual shared memory abstraction layer over distributed systems through software libraries, providing a multithreaded-like programming model for distributed application development. Compared to message-passing models, SDSM enables transparent cross-node memory access through a global address space, allowing developers to directly manipulate complex pointer-based data structures. This significantly reduces the complexity of developing and debugging distributed applications. In contrast to tightly coupled distributed shared memory systems, SDSM offers a larger shared memory space at a lower cost.

However, traditional SDSM systems have been limited by communication latency. Recently, advances in high-speed networking technologies, such as Remote Direct Memory Access (RDMA), have narrowed the latency gap between inter-node and intra-node communication, making SDSM systems a research hotspot once again in the fields of distributed computing and distributed storage.

Building on the domestic SDSM system JIAJIA, this paper designs and implements M-JIAJIA, an enhanced SDSM system supporting dual communication stacks (UDP and native RDMA). Leveraging modern multi-core server architectures and the diverse resource demands of parallel workloads, we propose the following optimizations:

\begin{enumerate}[label=\arabic*.]
    \item \textbf{Network Technology Optimization}: Introduce RDMA network technology to ensure high-speed and reliable communication between multiple hosts, reducing communication overhead between nodes;
    \item \textbf{Parallelization of Communication Tasks}: Introduce a multi-threaded coordination mechanism to partition communication tasks, ensuring efficient utilization of hardware resources;
    \item \textbf{Cross-Node Memory Access Optimization}: Design a remote prefetching mechanism that prefetches adjacent pages with restricted permissions when fetching remote pages, further reducing memory access latency for locality-sensitive applications;
    \item \textbf{Customizable Architecture}: The system separates the communication layer from system tasks, enhancing usability and scalability through a modular structure while maintaining performance, allowing for more flexible system configuration.
\end{enumerate}


The M-JIAJIA system was evaluated on several parallel test frameworks, including SPLASH-2 (LU, Water), NAS (EP, IS), and classical applications such as matrix multiplication (MM), Pi calculation, Traveling Salesman Problem (TSP), and equation solvers (Jacobi, SOR). Experimental results demonstrate that communication-intensive applications (TSP, MM, Water, LU, SOR) achieve performance improvements of 16\%-64\% on RDMA networks compared to traditional Ethernet. The prefetching mechanism improves performance for locality-sensitive applications on Ethernet (MM: 41\%, LU: 13\%, SOR: 12\%), while its effect is minimal in the RDMA network environment due to lower communication latency.

\KEYWORDS{SDSM, JIAJIA, Multithreading, RDMA}% 英文关键词

\pagestyle{enfrontmatterstyle}%
% \cleardoublepage\pagestyle{frontmatterstyle}%
