%---------------------------------------------------------------------------%
%->> Frontmatter
%---------------------------------------------------------------------------%
%-
%-> 生成封面
%-

\maketitle% 生成中文封面
\MAKETITLE% 生成英文封面
%-
%-> 作者声明
%-
\makedeclaration% 生成声明页
%-
%-> 中文摘要
%-
\intobmk\chapter*{摘\quad 要}% 显示在书签但不显示在目录
\setcounter{page}{1}% 开始页码
\pagenumbering{Roman}% 页码符号
软件分布式共享内存(SDSM)系统以软件库的形式在分布式系统上构建一个虚拟共享内存抽象层,为分布式应用开发提供类似多线程的编程范式。相较于消息传递模型,SDSM 提供的全局地址空间支持跨节点的透明内存访问,允许开发人员直接操作指针相关的复杂数据结构,显著降低了分布式程序的开发和调试复杂度。相较于紧耦合分布式共享内存系统,SDSM可以以低成本提供更大的共享内存空间。然而,传统 SDSM 系统长期受限于通信延迟问题。近年来,随着远程直接内存访问(RDMA)等高速网络技术的不断进步,跨节点访存和节点内访存的时延差距不断缩小,促使SDSM系统重新成为分布式计算和分布式存储领域的研究热点。

本文在国产软件分布式共享内存系统 JIAJIA 工作的基础上,设计并实现了支持 UDP 和原生 RDMA 的双通信栈软件分布式共享内存系统 M-JIAJIA。并根据现代多核服务器的硬件特性以及多样化并行负载的资源需求差异,提出以下优化方案:
\begin{enumerate}[label=\arabic*.]
    \item 通信任务并行化:引入多线程协同机制拆分通信流水线,实现无竞争消息的并行处理,充分利用多核计算资源;
    \item 跨节点访存优化:利用局部性原理设计远程预取机制,在远程取页时,预先取回相邻的页并给予受限制的权限,进一步降低了局部性强的应用的内存访问延迟;
    \item 可定制架构:整个系统采用分层设计原则,在保证性能的前提下,通过模块化结构提升系统易用性与可扩展性,支持通信协议、系统选项、优化技术等的灵活配置。
\end{enumerate}
%ep,is,lu,mm,jacobi,pi,sor,tsp

M-JIAJIA 系统在多个并行测试框架上的应用程序(SPLASH-2:LU、Water;NAS:EP、IS),以及一些经典并行应用如矩阵相乘(MM)、圆周率计算(PI)、旅行商问题(TSP),方程求解(Jacobi,SOR)上进行了测试。实验结果表明,对于通信密集的程序TSP、MM、Water、LU、SOR,相比于传统以太网,在RDMA 网络下运行的程序性能提升了22\times -58\times。远程预取机制在传统以太网上提升局部性强的应用性能 MM(41\times)、LU(13\times)、SOR(12\times),而在 RDMA 网络下,由于通信延迟较小,预取机制对性能影响并不明显。

\keywords{软件分布式共享内存系统, JIAJIA, 多线程, 远程直接内存访问}% 中文关键词
%-
%-> 英文摘要
%-
\intobmk\chapter*{Abstract}% 显示在书签但不显示在目录
    %- the current style, comment all the lines in plain style definition.
Software Distributed Shared Memory (SDSM) systems construct a virtual shared memory abstraction layer over distributed systems through software libraries, providing a multithreaded-like programming paradigm for distributed application development. Compared to message-passing models, SDSM offers a global address space that enables transparent cross-node memory access, allowing developers to directly manipulate complex pointer-based data structures, thereby significantly reducing the development and debugging complexity of distributed programs. Moreover, Unlike tightly coupled distributed shared memory systems, SDSM delivers a larger shared memory space at lower costs. However, traditional SDSM systems have long been constrained by communication latency issues. In recent years, advancements in high-speed networking technologies such as Remote Direct Memory Access (RDMA) have narrowed the gap between inter-node and intra-node communication, reigniting research interest in SDSM systems within the distributed computing field and distributed shared memory field.

Building upon the domestic SDSM system JIAJIA, this paper designs and implements M-JIAJIA, an enhanced SDSM system supporting dual communication stacks (UDP and native RDMA). To leverage modern multi-core server architectures and accommodate the heterogeneous resource demands of diverse parallel workloads, the following optimizations are proposed:

\begin{enumerate}[label=\arabic*.]{
    \item Parallelization of Communication Tasks: A multi-threaded cooperative mechanism splits communication pipelines to enable data-race-free message processing, fully utilizing multi-core computational resources.

    \item Cross-Node Memory Access Optimization: A locality-aware remote prefetching mechanism retrieves adjacent pages with restricted permissions during remote page fetches, effectively reducing memory access latency for applications with strong locality.

    \item Customizable Architecture: A layered design
    ensures performance while enhancing usability and extensibility through modular components, supporting flexible configurations of communication protocols, system options, and optimization techniques.
}
\end{enumerate}


The M-JIAJIA system was evaluated on multiple parallel benchmarks, including SPLASH-2 (LU, Water), NAS (EP, IS), and classical applications such as matrix multiplication (MM), Pi calculation, Traveling Salesman Problem (TSP), and equation solvers (Jacobi, SOR). Experimental results demonstrate that communication-intensive applications (TSP, MM, Water, LU, SOR) achieve 22×–58× performance improvements on RDMA networks compared to traditional Ethernet. The prefetching mechanism enhances performance for locality-sensitive applications on Ethernet (MM: 41×, LU: 13×, SOR: 12×), while the gains is trivial under RDMA network environment due to lower communication latency. 


\KEYWORDS{SDSM, JIAJIA, Multithreading, RDMA}% 英文关键词

\pagestyle{enfrontmatterstyle}%
\cleardoublepage\pagestyle{frontmatterstyle}%

%---------------------------------------------------------------------------%
