%---------------------------------------------------------------------------%
%->> Backmatter
%---------------------------------------------------------------------------%
\chapter[致谢]{致\quad 谢}\chaptermark{致\quad 谢}% syntax: \chapter[目录]{标题}\chaptermark{页眉}
%\thispagestyle{noheaderstyle}% 如果需要移除当前页的页眉
%\pagestyle{noheaderstyle}% 如果需要移除整章的页眉

首先,感谢张福新老师在这次毕业论文中对我耐心而专业的指导,和杨玉朋学长在实验过程中对我给予的细心的帮助,从而顺利完成本次毕业论文。
一篇论文不能代表我在计算机科学与技术方面的水平,更不应该止步于此,而是要学习张福新导师与杨玉朋学长不断学习的精神和锲而不舍的实验态度,
在计算机科学与技术领域开拓进取,在JIAJIA系统的基础上继续研究,成为该领域的人才。

其次,感谢给予我帮助的所有同学和相关科目的任课老师,是你们的帮助让我在本科学习期间成功渡过难关,完成四年的学业和今天的这篇毕业论文。
不论是我在本科学习计算机的过程中,还是我在毕业论文实验过程中遇到的各种技术难题与思考,在你们无偿的点拨下我最终才能够克服这些困难,

最后感谢我的家人,是他们多年来对我学业的支持才让我走到这一步,才使我得以顺利完成学业。


% \chapter{作者简历及攻读学位期间发表的学术论文与研究成果}

% % \textbf{本科生无需此部分}。

% \section*{作者简历:}

% \subsection*{基于RDMA新型网络的分布式虚拟内存系统 作者}

% 盛子轩,湖北省荆州市公安县人,中国科学院大学计算机科学与技术专业本科生。

% % \section*{已发表(或正式接受)的学术论文:}

% %  {
% %   \setlist[enumerate]{}% restore default behavior
% %   \begin{enumerate}[nosep]
% %       \item ucasthesis: A LaTeX Thesis Template for the University of Chinese Academy of Sciences, 2014.
% %   \end{enumerate}
% %  }

% \section*{申请或已获得的专利:}

% 《一种利用多线程加速加速基于UDP的软件共享分布式内存(SDSM)消息处理的设计方案》

% \section*{参加的研究项目及获奖情况:}

% 可以随意添加新的条目或是结构。

% \cleardoublepage[plain]% 让文档总是结束于偶数页,可根据需要设定页眉页脚样式,如 [noheaderstyle]
%---------------------------------------------------------------------------%
