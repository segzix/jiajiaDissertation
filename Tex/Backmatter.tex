%---------------------------------------------------------------------------%
%->> Backmatter
%---------------------------------------------------------------------------%
\chapter[致谢]{致\quad 谢}\chaptermark{致\quad 谢}% syntax: \chapter[目录]{标题}\chaptermark{页眉}
%\thispagestyle{noheaderstyle}% 如果需要移除当前页的页眉
%\pagestyle{noheaderstyle}% 如果需要移除整章的页眉

时光荏苒,四年本科生涯即将画上句号。在论文完成之际,我想借此机会向此期间给予我帮助和支持的师长、同窗和家人表达深深的感谢。

首先,我要向我的导师张福新研究员表达最诚挚的感谢。在论文写作过程中,张老师以其渊博的知识和严谨的学术态度,给予我细致的指导。
从选题到实验设计,从理论推导到论文修改,每一个环节都凝聚着张老师的心血。
特别感谢杨玉朋学长在实验阶段给予的无私帮助和经验指导,是在我们的共同协作下,才能成功完成M-JIAJIA系统的实践并撰写出这篇论文。
这篇论文的完成不是终点,而是新的起点。
我将以您们为榜样,在计算机科学与技术领域继续努力,特别是在JIAJIA系统的研究上勇攀高峰,为分布式系统的发展贡献自己的力量。

其次,我要衷心感谢计算机学院的各位任课老师。四年来,您们用每一次优秀的课堂点燃了我对计算机科学的热情,
并为我奠定了坚实的理论基础。同时,也要感谢同窗好友们的鼎力相助,无论是在课程学习中的疑难解答,
还是在毕业论文期间的讨论交流,你们的真知灼见总能让我豁然开朗,也将成为我人生中最宝贵的财富。

最后,谨以此文献给我最亲爱的家人。四年的求学之路,是您们无条件的支持与鼓励,才让我在迷茫时找到方向,并继续前行。


% \chapter{作者简历及攻读学位期间发表的学术论文与研究成果}

% % \textbf{本科生无需此部分}。

% \section*{作者简历:}

% \subsection*{基于RDMA新型网络的分布式虚拟内存系统 作者}

% 盛子轩,湖北省荆州市公安县人,中国科学院大学计算机科学与技术专业本科生。

% % \section*{已发表(或正式接受)的学术论文:}

% %  {
% %   \setlist[enumerate]{}% restore default behavior
% %   \begin{enumerate}[nosep]
% %       \item ucasthesis: A LaTeX Thesis Template for the University of Chinese Academy of Sciences, 2014.
% %   \end{enumerate}
% %  }

% \section*{申请或已获得的专利:}

% 《一种利用多线程加速加速基于UDP的软件共享分布式内存(SDSM)消息处理的设计方案》

% \section*{参加的研究项目及获奖情况:}

% 可以随意添加新的条目或是结构。

% \cleardoublepage[plain]% 让文档总是结束于偶数页,可根据需要设定页眉页脚样式,如 [noheaderstyle]
%---------------------------------------------------------------------------%
