\chapter{总结与展望}\label{chap:summary}{
    \section{本文工作总结}
    分布式共享内存作为一种重要的并行编程范式,在物理分布的系统上构建了一个逻辑上统一的地址空间。
    该范式融合了消息传递和共享内存的优点,不仅大大降低了分布式应用的开发复杂度,而且继承了分布式内存的高可扩展性。

    JIAJIA 是上世纪九十年代由计算技术研究所研发的国产软件 DSM 系统,
    基于域一致性内存模型的设计使得其在性能与编程复杂度之间获得良好的平衡,但由于网络技术不够成熟,导致实际运行时多节点通信往往成为整个系统的瓶颈。
    本文基于JIAJIA系统进行优化,开发出了M-JIAJIA系统。主要的优化有如下几点:
    \begin{enumerate}
        \item 重新进行UDP端口算法设计,并修改了多路复用机制与ack可靠通信机制,优化了UDP通信栈的构成;
        \item 将IO信号驱动单线程通信架构修改为了多线程通信架构,充分利用了硬件的性能;
        \item 分离通信层与核心逻辑层,使用循环消息队列作为两者之间的通信媒介,优化了系统的核心逻辑;
        \item 支持 UDP 和 RDMA 通信栈两种通信选择,利用RDMA网络技术,成功降低了系统的通信开销,提升了系统的性能。
    \end{enumerate}

    \section{未来研究方向}

    未来研究工作将围绕以下几个方面展开:

    \begin{enumerate}
        \item \textbf{RDMA原语深度优化}:进一步探索RDMA read/write原语在M-JIAJIA系统中的实现,探索原子操作等高级特性的集成方案,
              提升跨节点内存访问效率,尽可能替代RDMA recv/send原语。

        \item \textbf{系统接口扩展}:完善JIAJIA对外接口体系,增强开发者友好性,尽可能使JIAJIA成为一个可靠且能大规模部署的分布式共享内存系统。
              同时建立标准化接口规范,提升系统可扩展性。

        \item \textbf{分布式内存数据库构建}:基于JIAJIA构建高性能分布式内存数据库系统,充分利用RDMA网络的快速传输特性。

        \item \textbf{一致性模型增强}:扩展JIAJIA内存一致性模型,支持多粒度cache一致性协议。
              同时继续探索宿主迁移等适合JIAJIA系统性能优化的方案,尽可能免除不必要的通信,对通信瓶颈进行优化。
    \end{enumerate}
}