\chapter{总结与展望}\label{chap:summary}{
    \section{本文工作总结}
    分布式共享内存作为一种重要的并行编程范式,在物理分布的系统上构建了一个逻辑上统一的地址空间。
    该范式融合了消息传递和共享内存的优点,不仅大大降低了分布式应用的开发复杂度,而且继承了分布式内存的高可扩展性。

    JIAJIA 是上世纪九十年代由计算技术研究所研发的国产软件 DSM 系统,
    基于域一致性内存模型的设计使得其在性能与编程复杂度之间获得良好的平衡,但由于网络技术不够成熟,导致实际运行时多节点通信往往成为整个系统的瓶颈。
    本文基于JIAJIA系统进行优化,开发出了M-JIAJIA系统。主要的优化有如下几点:
    \begin{enumerate}
        \item 重新进行UDP端口算法设计,并修改了多路复用机制与ack可靠通信机制,优化了UDP通信栈的构成;
        \item 将IO信号驱动单线程通信架构修改为了多线程通信架构,充分利用了硬件的性能;
        \item 分离通信层与核心逻辑层,使用循环消息队列作为两者之间的通信媒介,优化了系统的核心逻辑;
        \item 支持 UDP 和 RDMA 通信栈两种通信选择,利用RDMA网络技术,成功降低了系统的通信开销,提升了系统的性能。
    \end{enumerate}

    \section{未来研究方向}

    \begin{enumerate}
        \item RDMA read write 原语的进一步实现;
        \item jiajia对外接口扩展;
        \item 开发基于jiajia的分布式内存数据库;
        \item jiajia内存cache一致性模型的进一步扩展。
    \end{enumerate}
}